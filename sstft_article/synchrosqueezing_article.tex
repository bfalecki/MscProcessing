\documentclass[conference]{IEEEtran}
\IEEEoverridecommandlockouts
% The preceding line is only needed to identify funding in the first footnote. If that is unneeded, please comment it out.
\usepackage{cite}
\usepackage{amsmath,amssymb,amsfonts}
\usepackage{algorithmic}
\usepackage{graphicx}
\usepackage{textcomp}
\usepackage{xcolor}
\usepackage[inkscapeformat=png]{svg}
\usepackage{float}
\usepackage{amssymb}
\usepackage[utf8]{inputenc}
\usepackage[T1]{fontenc}
\usepackage[polish,english]{babel}
%\usepackage{polski}
\usepackage[nottoc]{tocbibind}
\usepackage{comment}
\def\BibTeX{{\rm B\kern-.05em{\sc i\kern-.025em b}\kern-.08em
    T\kern-.1667em\lower.7ex\hbox{E}\kern-.125emX}}

\usepackage[inkscapeformat=png]{svg}

\addto\captionspolish{\renewcommand{\figurename}{Fig.}}
\addto\captionspolish{\renewcommand{\refname}{References}}
\addto\captionsenglish{\renewcommand{\figurename}{Fig.}}
\renewcommand{\figurename}{Fig.}
\renewcommand{\refname}{References}  % Dla artykułów (article)



\begin{document}

\title{Decomposition of Radar-Measured Vital Signs Using Vertical Synchrosqueezing
}

\author{\IEEEauthorblockN{Bartosz Falęcki, Karol Abratkiewicz}
\IEEEauthorblockA{\textit{Institute of Electronic Systems} \\
\textit{Warsaw University of Technology}\\
00-665 Warsaw, Poland \\
bartosz.falecki.stud@pw.edu.pl}
}

\maketitle

\begin{abstract}
This paper focuses on the processing of radar measurement-based vital signs using vertical synchrosqueezing to increase the resolution of the time-frequency distribution and component separation. The vertical synchrosqueezing method was used to decompose the signal into breathing and heartbeat components. In the first step, the component related to respiration is extracted from the signal. Then, it is removed to obtain a waveform corresponding to the heart activity with reduced noise and the influence of chest movement. Finally, the heart rate is estimated using the spectral analysis of the instantaneous signal strength.
\end{abstract}

\begin{IEEEkeywords}
radar, heartbeat, synchrosqueezing, vital signs
\end{IEEEkeywords}

\section{Introduction}
Microwave-assisted monitoring of vital signs is a concept that dates back to 1975 when J.C. Lin \cite{breath1975} managed to measure human and also rabbit's breath-related displacements.
Then, in 1977, I. Arai et al. \cite{japan1977} published work under a successful measurement of breath and heartbeat.
However, in recent years, this idea has been gaining more and more interest due to the non-contact measurement. Since then, countless publications have been published describing both hardware solutions and signal processing. Various types of radars are used for this purpose, including frequency modulated continuous wave (FMCW) \cite{fmcw1}, but also impulse radio ultra-wideband \cite{uwb1} or continuous-wave Doppler radar \cite{cw1} in a wide range of carrier frequencies from several GHz \cite{2ghz} to over 100 GHz \cite{120ghz}.
For processing, the signal phase demodulation is typically used, which is directly related to the displacement of the observed surface. For this purpose, there are advanced algorithms for decomposition and estimation of vital parameters, some of which use, e.g., empirical model decomposition \cite{emd1} or even deep learning \cite{DL}.
 

Less popular than phase demodulation is the use of time-frequency (TF) analysis, perhaps due to its computational complexity. However, it is a powerful representation that can be successfully used in real-time applications \cite{realtime1}.
Measuring human vital signs using radar is challenging because the only information used for estimation is the micro-Doppler signature related to the displacement of the chest due to breathing and heart activity.
The superposition of these two components forms a signal that has both a slowly varying component associated with respiration and cyclic, high-frequency heart-related oscillations.
To obtain a faithful representation of the Doppler frequency changes for this type of signal, high resolution of both time and frequency is necessary.
Nevertheless, typical TF methods often have unsatisfactory trade-offs between time and frequency resolution or are burdened with non-negligible interferences \cite{reas}.
To increase the resolution of the TF plane, some techniques can be used to shift the distribution points towards higher energy locations to concentrate the image.
The origin of TF representation concentration techniques is almost as old as the concept of microwave measurement of vital signs.
A technique similar to nowadays called reassignment was first described in 1976 \cite{kodera} under the name of the modified moving window method.
Since then, many different concentration techniques have been developed, and in some cases, a thorough mathematical background has been formulated \cite{sync}.
With this kind of approach, it is possible to significantly improve the resolution of the TF distribution in one dimension while maintaining satisfactory resolution in the other dimension or to concentrate both dimensions simultaneously.
The first method, in which points are moved in only one dimension, is called synchrosqueezing \cite{sync}, and its use allows for fully recovering the original signal from the concentrated TF plane.
Synchrosqueezing can be divided into the vertical type, where the points are moved only along the frequency axis, and the horizontal type, where the points are moved only along the time axis.
The second method, in which points can be moved in both axes simultaneously, is an above-mentioned reassignment \cite{reas}.
In this case, the original signal reconstruction is no longer possible from the processed TF plane, but reassignment possibly allows for a better concentration than synchrosqueezing \cite{sync}.
 
\section{Vertical synchrosqueezing}

Here is a brief introduction to the first-order vertical synchrosqueezing method, which is used in this paper to decompose the vital signs' radar signal. First, let us consider the short-time Fourier transform (STFT) of signal $f$ according to the following definition \cite{sync}:
\begin{equation}
V_{f}^{g} (t,\omega) = \int_{\mathbb{R}} f(\tau) g^{\ast}(t-\tau)e^{-i\omega \tau}\mathrm{d}\tau ,
\end{equation}
where $g^{\ast}(t)$ is the complex conjugate of the window $g(t)$ which is typically a Gaussian function, and $i=\sqrt{-1}$.
To estimate the exact instantaneous frequency of a supposed signal component in the vicinity of a point $(t,\omega)$ whose energy is spread out, causing an increase in energy at that point, an instantaneous frequency estimator is introduced, which can be expressed by the formula \cite{sync}
\begin{equation}
\hat{\omega}_{f}(t,\omega)=\omega-\Im \left\{ V_{f}^{g^{\prime}}(t,\omega)\over V_{f}^{g}(t,\omega)\right\},
\end{equation}
where $\Im \left\{\cdot\right\}$ is an imaginary part of the complex value, and $V_{f}^{g^{\prime}}(t,\omega)$ is the STFT of $f$ with the derivative of the window function $g^{\prime}(t)=\frac{\mathrm{d}g(t)}{\mathrm{d}t}$. $\hat{\omega}_{f}(t,\omega)$ can be interpreted as the local center of gravity for energy in the frequency axis.
Synchrosqueezed STFT (SSTFT) is then obtained by redistribution of each STFT point to the estimated instantaneous frequency $\hat{\omega}_{f}(t,\omega)$ according to the following rule \cite{sync}:
\begin{equation}
T_{f}^{g}(t,\omega) ={1\over 2\pi g(0)^{\ast}} {\int}_{\mathbb{R}} V_{f}^{g}(t,v)e^{i\omega t} \delta (\omega -\hat{\omega} _{f} (t,v)) \mathrm{d}v,
\end{equation}
where $\delta (\omega)$ is the Dirac delta distribution and $g(0)\ne0$ is the window value at its origin.
Therefore, the individual distribution points are moved only along the frequency axis, which allows for the ideal reconstruction of the original signal $f$, similar to the case of the ordinary STFT. The inverse transform is done by integrating the distribution along the frequency axis, which is expressed by the equation \cite{sync}
\begin{equation}
f(t)={1 \over 2\pi g(0)^{\ast}}{\int}_{\mathbb{R}}T_{f}^{g}(t,\omega)e^{i\omega t} \mathrm{d}\omega.
\end{equation}
The distribution $T_{f}^{g}(t,\omega)$ can be previously multiplied by a binary mask to extract a particular signal component.


\section{Measurement setup}
The experimental radar recording of vital signs was obtained using the FMCW XY-DemoRad system \cite{demorad}.
The device transmitted a high-frequency wave of $121.16$ GHz with a bandwidth of $5$~GHz.
The relaxed person was in a motionless sitting position with a $30$~cm chest-to-antenna distance.
The radar antenna was pointed directly at the chest without any obstructions.
The entire recording lasts $10$~s at a baseband signal sampling rate of $1$~MHz and a pulse repetition rate of $5000$~pulses per second.
The recording was saved to the hard disk and then post-processed in the MATLAB environment.

\section{Processing}

The signal processing scheme is shown in Fig.~\ref{fig:processing}.
The first step after selecting the appropriate range cell from the range-time map is to calculate the SSTFT. In order to obtain an acceptable time resolution, a Gaussian window with full width at half maximum (FWHM) of about $0.11$~s is used.
Then, the strongest TF ridge is detected using a penalized forward-backward greedy algorithm (MATLAB \texttt{tfridge} function), which mainly follows the respiratory component.
If the energy of the respiratory component is too widely distributed on the TF plane, concentration can cause splitting ridges. If this occurrence is significant, several major ridges can be detected.
Based on the detected TF ridge, a binary mask is generated in its certain neighborhood, which mainly contains the slowly varying respiratory component in the TF plane. In this particular case, the bilateral width of the neighborhood in the frequency axis is approximately $15$~Hz.
If several ridges are detected, the resulting mask can be computed by a logical sum of each ridge-related mask.
The reconstruction of the respiratory echo signal is achieved using the inverse SSTFT according to (4) with a binary mask applied to filter out unnecessary higher components.
To obtain the echo signal of cardiac activity with the respiratory component filtered out, the same transformation was applied, but the binary mask was inverted.
After reconstruction, the instantaneous signal power of the heartbeat component is calculated using a windowed root mean square (RMS) envelope with the window width of $0.2$~s.
Specifically, the envelopes are calculated separately for the real and imaginary parts and then combined into one by averaging.
Next, the constant component is removed from the envelope by subtracting the mean value.
Finally, the spectrum analysis of the envelope is computed to estimate the heart rate.


\begin{figure}
 \centering
   \includegraphics[width=\linewidth]{plot/sps_processing.drawio.png}
   \caption{Processing scheme of the signal decomposition.}
    \label{fig:processing}
\end{figure}


\section{Results}

The radar echo of measured vital signs is shown in Fig.~\ref{fig:raw}. The recording shows two slow respiratory cycles. The signal shows changes in Doppler frequency related to respiratory-related forward and backward movement of the chest. At first glance, no heart-related vibrations are visible, but TF analysis reveals them as cyclic changes in energy density along the main echo line Fig.~\ref{fig:stft}. In the TF plane, it is also much easier to see the shape of the frequency modulation waveform related to breathing.
Fig.~\ref{fig:sstft} shows a concentrated spectrogram using first-order vertical synchrosqueezing. Almost all of the signal energy is focused on the main echo line, which facilitates further decomposition.
Fig.~\ref{fig:hb_sstft} shows the synchrosqueezed TF plane with the respiratory component cut out by using an inverted mask following the detected TF ridge.
Fig.~\ref{fig:breath_signal} and Fig.~\ref{fig:hb_signal} show the reconstructed signal using the inverse transform when the applied mask is non-inverted and inverted, respectively. The first case corresponds to the respiratory component, while the second is the echo related to cardiac activity. In the latter case, repetitive high-frequency oscillations that correspond to individual heart cycles are clearly visible.
It is worth noting that the echo related to the heart is so weak compared to the respiration that there is almost no difference between the raw signal and the reconstructed respiration signal (cf. Fig.~\ref{fig:raw} and Fig.~\ref{fig:breath_signal}).
The RMS envelope of the reconstructed heartbeat component is shown in Fig.~\ref{fig:env}.
The envelope window was optimized to obtain a prominent heart rate peak in the spectrum analysis (Fig.~\ref{fig:spectrum}). The main peak located at $1.27$~Hz gives an estimated heart rate of $76$~BPM.
To view the cardiac cycles at a higher temporal resolution, it would be more appropriate to use a shorter window, but this would result in an increase in harmonics in the frequency analysis.



\begin{figure}
 \centering
   \includegraphics[width=\linewidth]{plot/breath1_rank1_raw_signal.eps}
   \caption{Raw measured vital signs.}
    \label{fig:raw}
\end{figure}

\begin{figure}
    \centering
    \includegraphics[width=\linewidth]{plot/breath1_rank1_stft.png}
    \caption{STFT spectrogram of the signal.}
    \label{fig:stft}
\end{figure}


\begin{figure}
    \centering
    \includegraphics[width=\linewidth]{plot/breath1_rank1_sstft.png}
    \caption{Signal spectrogram after the first-order vertical synchrosqueezing.}
    \label{fig:sstft}
\end{figure}

\begin{figure}
    \centering
    \includegraphics[width=\linewidth]{plot/breath1_rank1_heartbeat_sstft.png}
    \caption{Synchrosqueezed energy distribution with the inverted binary mask applied to filter out respiratory component. The detected TF ridge is marked with a blue line.}
    \label{fig:hb_sstft}
\end{figure}

\begin{figure}
 \centering
   \includegraphics[width=\linewidth]{plot/breath1_rank1_breath_signal.eps}
   \caption{Breath-related echo signal reconstructed from the synchrosqueezed TF plane with the mask applied.}
    \label{fig:breath_signal}
\end{figure}

\begin{figure}
 \centering
   \includegraphics[width=\linewidth]{plot/breath1_rank1_heartbeat_signal.eps}
   \caption{Heartbeat-related echo signal reconstructed from the synchrosqueezed TF plane with the inverted mask applied.}
    \label{fig:hb_signal}
\end{figure}


\begin{figure}
    \centering
    \includegraphics[width=\linewidth]{plot/breath1_rank1_env.eps}
    \caption{RMS envelope of the heartbeat-related echo signal.}
    \label{fig:env}
\end{figure}


\begin{figure}
 \centering
   \includegraphics[width=\linewidth]{plot/breath1_rank1_hb_spectrum.eps}
   \caption{Spectrum of the heartbeat signal envelope.}
    \label{fig:spectrum}
\end{figure}


\section{Conclusions}
In summary, synchrosqueezing is an effective method for decomposing radar vital signs signals. The echo signal related to the heartbeat has been reconstructed, and the heart rate was estimated.
The advantage of TF analysis over phase demodulation is that it is a natural way to represent a signal consisting of many components, which allows for easy decomposition. This is particularly important in radar applications, where many independent signals may be observed simultaneously. In addition, the TF representation contains information about the energy distribution of the signal, while the phase waveform alone may often be not reliable enough. TF-plane can also be used as input to efficient image processing algorithms based on neural networks.

\bibliographystyle{unsrt}
\bibliography{references}

\end{document}
