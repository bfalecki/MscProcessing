\subsection{Literature Review on Vital Signs Measurement}
Measurement of human vital signs using radar is a widely studied concept due to its non-contact and attractive manner of operation. The genesis of the idea reaches the early 1970s, and its popularity is still increasing simultaneously with the development of the related radar devices.


\subsubsection{General Measurement Principle}
The general principle of measurement is similar for all types of radar architectures. Radar can detect human vital signs thanks to the micro-Doppler phenomenon in the radio waves reflected from the human body.
Micro-Doppler signatures are the frequency changes of the reflected signal due to the specific movements of the  observed surface.
In this case, respiratory-related displacements and heartbeat-related vibrations of the human chest generate the observable response which is then extracted in processing.

\subsubsection{Wave Reflectance of Human Body}