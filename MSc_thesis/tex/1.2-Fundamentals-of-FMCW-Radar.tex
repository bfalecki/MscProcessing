\subsection{Fundamentals of FMCW Radar} 
% general description, Advantages and Disadvantages
In this section, the FMCW radar architecture is explained in detail to provide a brief background for further considerations on signal processing and system limitations. In particular, the operation of the system will be described in terms of data collection and use of signal features to determine the speed and distance of targets. At the end of the section, the presented methods for extracting micro-Doppler signatures from FMCW radar data will be briefly discussed.


\subsubsection{System Composition}
The general block diagram of the FMCW radar system is shown in Fig.\ref{fig:FMCW_scheme}. 
In reality, the radar structure is slightly more complex, but for explanatory purposes, this diagram reflects the principle of operation of the system quite well.
First, the linear frequency modulation signal needs to be generated. For this reason, a sawtooth signal generator in connection with a voltage controlled oscillator can be used \cite{FMCW_implementation}.
The next step is to divide the signal power into the transmission path and the reference path. The transmission path goes through a power amplifier, ending with the transmit antenna. The signal emitted in a form of radio waves can then reflect from the surrounding targets and reach the receive antenna.
After low-noise amplification of the reflected signal, it is mixed with the reference signal for the purpose of analog shift to the baseband, making the digitization process possible. The mixer output is a signal consisting of the sum and difference of the frequencies of the transmitted and received signals. The mathematical background of this operation will be presented in more detail in the next section. Only the difference-frequency component of the signal is useful, while the sum-frequency component must be removed. For this purpose, a low-pass filter is used to prevent aliasing \cite{FMCW_aliasing}.
The next step is to adjust the signal's power for the analog-to-digital converter input using another amplifier. The digitized signal can then be stored or processed with built-in/external signal processing unit.

Additionally, an FMCW radar is typically equipped with two signal paths, one of them being in a quadrature and the other in phase.
For this purpose, after the oscillator output, two power dividers can be used in series, producing two reference signals. One of the reference signals is then phase-shifted by 90 degrees, creating the quadrature component. The second signal, called the in-phase component, remains unchanged.
Both reference signals are independently mixed with the same received signal, so the signal in the receiving path must first be split into two paths. After the mixer outputs, the acquisition path for both signals remains independent and corresponds to the one shown in Fig.\ref{fig:FMCW_scheme}.

The in-phase and quadrature components are then saved as a single complex-valued signal, referred to as IQ.
Using IQ acquisition allows for a doubling of the signal bandwidth, as a single complex sample is composed of two real values. The great advantage of this solution is that the impact  of unfavorable phase relationships between the transmitted and received signals is reduced. Using only one component would make the extracted signal phase less resistant to noise. Furthermore, the phase would be susceptible to drops in displacement sensitivity at certain distances \cite{Alizadeh2019}.


\begin{figure}
  \centering
 \includesvg[width=\linewidth,inkscapelatex=true]{"img/FMCW_scheme.drawio.svg"}
  \caption{Simplified block diagram of FMCW radar.}
  \label{fig:FMCW_scheme}
\end{figure}

\subsubsection{Waveform Shape}
To explain what exactly happens to the signal when it interacts with the target and is mixed with the reference, this section describes this path in mathematical language.

\paragraph{Transmitted Signal}
First of all, the transmitted signal can be considered as a waveform with linear frequency modulation, cyclically repeating every Pulse Repetition Interval $T_\mathrm{PRI}$.
A single $n$-th period of the signal, i.e., for the time interval $t \in \left[n T_\mathrm{PRI} , (n+1) \cdot T_\mathrm{PRI} \right)$, is called a chirp, and can be expressed as \cite{Alizadeh2019}

\begin{equation}
x_t(t) = A_t \exp\left[\mathrm{j}\left(2 \uppi f_\mathrm{min} t + \uppi K t^2 + \upvarphi_n(t)\right)\right],
\tag{3}
\end{equation}

where:  
\begin{itemize}
    \item $A_t$ – amplitude of the transmitted signal,
    \item $B$ – modulation bandwidth, expressed in Hz,
    \item $f_\mathrm{min}$ - starting chirp frequency, $f_\mathrm{min} = f_0 - B/2$, where $f_0$ is the carrier (center) frequency,
    \item $K$ – modulation rate, $K = \frac{B}{T_{\mathrm{PRI}}}$,
    \item $\upvarphi_n(t)$ – nonlinearity of the oscillator characteristic and phase noise.
\end{itemize}

Variations of the above signal shape are also common, in which, for example, a triangular frequency modulation is used instead of a sawtooth modulation.

\paragraph{Received Signal}
% TO DO: ensure about the considerations on the Doppler frequency
If we assume that the received signal is ideally a transmitted signal reflected from a moving target, it will contain a Doppler frequency shift and some delay due to the propagation time.
Fortunately, true Doppler shift within a single chirp can be neglected because it starts to matter only with a set of consecutive chirps.
Therefore, the corresponding $n$-th chirp of the received signal for $t \in \left[n T_\mathrm{PRI} , (n+1) \cdot T_\mathrm{PRI} \right)$ can be approximated as
\begin{equation}
x_r(t,n) = A_r \exp\!\left[
    \mathrm{j}\!\left(
        2\uppi f_\mathrm{min}(t - t_d(n))
        + \uppi K (t - t_d(n))^2
        % + 2\uppi f_d (t - t_d) % this phase component is included in the first term
        + \upvarphi_n(t - t_d(n))
    \right)
\right],
\end{equation}

where $A_r$ is the amplitude of the received signal, and $t_d(n)$ is the time delay between the transmitted and received signals for the $n$-th chirp. The time delay can be assumed to be constant for a single chirp \cite{Alizadeh2019}, and it can be expressed as $t_d(n) = 2R(n)/c$, where $R(n)$ is the radar-to-target distance for the $n$-th chirp \cite{Scherr2017}.

% The simplified Doppler frequency change introduced there as a direct modulation by a constant $f_d$ is not an exact representation of reality. As the instantaneous frequency of the signal changes in time according to the modulation, the Doppler frequency is also affected by this change. However, if we assume  $f_0 \gg B$, the Doppler frequency can be assumed to be \cite{FMCW_implementation}
% \begin{equation}
%     f_d = - \frac{2v f_0}{c},
%     \label{eq:doppler_freq}
% \end{equation}
% where $v$ is the target's radial velocity with respect to the radar ($v > 0$ means moving away and $v < 0$ means moving closer). (\ref{eq:doppler_freq}) also uses a simplification related to neglecting relativistic effects, which is applicable for $v \ll c$.

\paragraph{Intermediate Frequency Signal}
In this paragraph, the result of mixing the received and transmitted signals, after eliminating higher components using a low-pass filter, is briefly described. It is called the demodulated signal or the intermediate frequency signal, as it no longer has frequency modulation and is shifted to baseband.
The signal has a frequency equal to the instantaneous difference between the frequencies of the transmitted and received signals.
The intermediate frequency signal can be approximated by \cite{Ding2016} (quoted after \cite{Alizadeh2019})
\begin{equation}
    x_i(t,n) = x_t^*(t) x_r(t)\approx A_t A_r
    \exp\!\left[
        \mathrm{j}\!\left(
            2\uppi K t_d t
            + 2\uppi f_\mathrm{min} t_d
            %- 2\uppi f_d (t - t_d) % this phase component is the same as the second one
            + \Delta\upvarphi_n(t)
        \right)
    \right],
    \label{eq:demod_signal}
\end{equation}
and can be rewritten as
\begin{equation}
    x_i(t,n) \approx A_t A_r
    \exp\!\left[
        \mathrm{j}\!\left(
             \frac{4 \uppi K R(n)}{c} t
             +\frac{4 \uppi f_\mathrm{min} R(n)}{c} 
            + \Delta\upvarphi_n(t)
        \right)
    \right],
    \label{eq:demod_signal2}
\end{equation}

where $\Delta \upvarphi_n(t) = \upvarphi_n(t - t_d) - \upvarphi_n(t)$ is the phase noise remaining after the mixing operation. The amplitude of $\Delta \upvarphi_n(t)$ is smaller the closer the observed target is, because the noise $\upvarphi_n(t - t_d)$ is more correlated with $\upvarphi_n(t)$ \cite{Budge1993}.
What is most important, (\ref{eq:demod_signal2}) contains the information about the range and velocity of the target, which will be the subject of the next section.
% The term $- 2\uppi f_d (t - t_d)$ responsible for the Doppler frequency shift does not originally appear in \cite{Alizadeh2019}. It becomes significant for the set of received signal periods, consistently causing increments in their phase due to the target's velocity.
The intermediate frequency signal can now be digitized, as it has a sufficiently low bandwidth, which is typically much lower than the bandwidth of the transmitted or received signals.

\subsubsection{Range and Velocity Determination} 
% formulas for range and velocity
If we consider an intermediate frequency signal as (\ref{eq:demod_signal2}), it contains a so called beat frequency $f_b = 2 K R(n)/c$ within a single chirp, which is linearly proportional to the range $R(n)$.
Also, a phase component $\uptheta(n) = 4 \uppi f_{min} R(n) / c$, proportional to the range $R(n)$, is present, which is assumed to be constant for a single chirp, but it can change over a set of chirps.
These features are directly used for the estimation of the target's range, displacement, and velocity.
The formula for the target's range based on a single chirp can be simply derived as
\begin{equation}
    R = \frac{c f_b}{2K}.
    \label{eq:range}
\end{equation}
This means that the beat frequency of the signal corresponding to a single chirp needs to be extracted to obtain the target's range.
%% fast time fft
For this purpose, a Fourier transform is commonly used, and in this context, it is called a \emph{range compression}. Every acquired chirp is processed independently in a similar way, which is usually referred to as a Fourier transform in a fast time dimension. The result is a set of range profiles, called a range-time map, containing frequency peaks corresponding to the distances of nearby objects depending on time. The range-time map is a two-dimensional function, in which the first dimension is a range axis, $r$, and the second dimension is a chirp number, $n$, called a slow time axis. Therefore, it can be expressed as
\begin{equation}
    X(r,n)=\int \limits_{-\infty}^{\infty} x_i(t, n) \mathrm{e}^{-\mathrm{j} \upomega_r t }\mathrm{d}t,
    \label{eq:range_time}
\end{equation}
where $\upomega_r = 4 \uppi Kr/c$.

Unlike range, velocity cannot be obtained using only a single chirp, as the phase relations in a set of consecutive chirps are needed to extract it. For this purpose, a signal consisting of many chirps is considered for the target's echo repetitively occurring at a certain distance.
Usually, two more assumptions are made to extract the target velocity from the signal within a specific time period in a simple way. First one is that the total change in target distance over the selected time period does not cause the echo to shift more than the range resolution.
Also, a nearly constant velocity is assumed, which means that, for the considered time period, the velocity does not change more than the velocity resolution corresponding to this period.
% slow time fft
When the above assumptions are met, the signal in a slow time dimension, extracted from the range-time map at a nearly constant range $R$, can be approximated as
\begin{equation}
    X_R(n) = A_{X_R}(n) \exp 
    \left[
        \mathrm{j} 
        \left(
            \uptheta (R)
            +\frac{4 \uppi f_\mathrm{min} T_\mathrm{PRI} V  n }{c} 
        \right)
    \right],
    \label{eq:slow_time}
\end{equation}
where:  
\begin{itemize}
    \item $A_{X_R}(n)$ - the range-time map amplitude at the range $R$, 
    \item $\uptheta (R)$ -  a phase offset component related to the range,
    \item $V$ - the target velocity.
\end{itemize}
The slow time signal (\ref{eq:slow_time}) contains a so called Doppler frequency $f_d = 2f_{min} V/c$, which needs to be extracted for the velocity estimation. As in the case of range estimation, the Fourier transform is used for this purpose. Therefore, the entire range time map can be processed by calculating a discrete time Fourier transform in a slow time dimension. The result is commonly called a range-Doppler map, as its first dimension is a range axis, $r$, and the second one is a Doppler velocity axis, $v$. It can be therefore expressed as
\begin{equation}
    Y(r,v) = \sum \limits_{n=-\infty}^{\infty}   
    X(r, n) \mathrm{e}^{-\mathrm{j} \upomega_v n },
    \label{eq:range_doppler}
\end{equation}
where $\upomega_v = 4 \uppi f_\mathrm{min} T_\mathrm{PRI} V / c$.
The range-Doppler map $Y(r,v)$ can already be considered as a detector of targets with certain movement features. If there is a target present at a certain range with a certain velocity relative to the radar, the range-Doppler distribution amplitudes of the corresponding area should be higher than in other areas. Of course, more targets can be detected simultaneously as long as their ranges differ by more than the range resolution or their velocities differ by more than the velocity resolution.

% maximum unambiguous range and velocity
In practice, an efficient implementation called Fast Fourier Transform (FFT) is used to calculate the Fourier transforms in (\ref{eq:range_time}) and (\ref{eq:range_doppler}).
Due to the finite sampling frequency $f_s$ of the demodulated signal $x_i(t,n)$, components with frequencies higher than $f_s$ are not possibly to observe without ambiguity. For this reason, the FMCW radar has a maximum unambiguous range, which is the longest distance at which it can observe targets without ambiguities related to the Nyquist condition. Substituting $f_s$ for $f_b$ into (\ref{eq:range}), the maximum unambiguous range can be expressed as
\begin{equation}
    R_\mathrm{max} = \frac{c f_s}{2K}.
    \label{eq:max_range}
\end{equation}
A similar limitation occurs for velocity and it is related to a pulse repetition interval, because in this context, the slow time signal is sampled once per chirp. Therefore, the maximum unambiguous velocity for a constant pulse repetition interval $T_\mathrm{PRI}$ can be derived from (\ref{eq:slow_time}) as
\begin{equation}
    V_\mathrm{max} = \frac{c}{4 f_\mathrm{min} T_\mathrm{PRI}}.
\end{equation}
It is worth noting that targets can also be observed unambiguously at negative velocities, i.e., the total interval of unambiguous velocity is $\left[-V_\mathrm{max}, V_\mathrm{max}\right)$. In the case of range, only positive values are considered, giving the unambiguous range interval of $\left[0, R_\mathrm{max}\right)$.

% (Optional) Error of range measurement for the target with higher velocity (this is solved when using triangular modulation) ?? is it a real error or maybe we measure the range in the middle of the chirp? Or a combination of these?


\subsubsection{Micro-Doppler Extraction}
Besides range-Doppler processing, there are other methods used for the extraction of unique motion features. When the target generates echoes containing Doppler frequency components significantly varying in time due to its internal movement, the related footprints left in the signal are called micro-Doppler signatures. These types of movements include, for example, vibrations of an engine, rotations of drone blades, moving human limbs, or displacements of the human chest due to breath and heartbeat. This section presents two leading techniques for extracting micro-Doppler signatures: phase extraction and time-frequency analysis.
\paragraph{Phase Extraction} %  (formula transforming displacement to phase)
Taking advantage of certain assumptions, it is possible to measure the exact displacement waveform of the target based on the phase changes between consecutive chirps.
If the reflection originates mostly from a single target, the phase of the slow time signal (\ref{eq:slow_time}) corresponds to the relative displacement $r_d(n)$ with respect to $R$, according to the equation
\begin{equation}
    \Phi(n) = \uptheta (R)
            +\frac{4 \uppi f_\mathrm{min}}{c}  \cdot r_d(n)
            .
    \label{eq:phase}
\end{equation}
From (\ref{eq:phase}), displacement can be derived as
\begin{equation}
    r_d(n) = \frac{c\left(\Phi(n) - \uptheta (R)\right)}{4 \uppi f_\mathrm{min}},
\end{equation}
which means that for the precise determination of the displacement waveform $r_d(n)$, only the signal phase extraction is needed.
One of the problems is that the measured phase is wrapped within the interval $[0, 2\pi)$. For this reason, the raw phase waveform needs to be unwrapped to extract a valid displacement waveform. % mention common algorithms
% This causes the noise impact to be additionally problematic.  %%% a source ????
Therefore, in order to correctly extract the phase waveform, in addition to the approach using the $\tan^{-1}$ function, the Differentiate and Cross-Multiply algorithm (DACM) is popular with its variations \cite{dacm}.
Naturally, obtained displacement waveform can be further differentiated to extract the instantaneous velocity of the observed surface, which is a commonly used method to observe human breath and heart activity.

\paragraph{Time-Frequency Analysis} %  (STFT, TF-resolution)
When the micro-Doppler signature is composed of echoes from multiple, differently moving scatterers, their corresponding signal components add up, producing a more complex footprint.
In such cases, simply determining the signal phase does not allow for the separation of the displacement functions of individual scatterers.
To extract the micro-Doppler signature contained in this type of signal, time-frequency methods can be used, i.e., transformations that allow for the representation of changes in the signal spectrum over time.
Changes in the Doppler frequency spectrum can help distinguish scatterers when they occur at different instantaneous velocities.
Time-frequency methods involve comparing a signal with basis functions that are concentrated in both frequency and time. One of the most commonly used transformations of this type is the Short-Time Fourier Transform (STFT). The STFT transform for signal $s(\uptau)$ is defined by the following equation \cite{qian1996joint}
\begin{equation}
    \mathrm{STFT}(t,\upomega)
    = \int s(\uptau)\,\upgamma(\uptau - t)\,
    \exp\!\left(-\mathrm{j}\upomega\uptau\right)\,\mathrm{d}\uptau,
\end{equation}
where $\upgamma(t)$ is a time window function, typically a Gaussian function.
The result of the STFT transformation can be presented in a spectrogram, a 2-dimensional energy distribution depending on time and frequency:
\begin{equation}
    \mathrm{SP}(t,\upomega)
    = |\mathrm{STFT}(t,\upomega)|^2.
\end{equation}
The STFT transform is burdened with a certain time resolution $\Delta t$ and frequency resolution $\Delta \upomega$, which are closely related. They result from the width of the time window and its frequency band, respectively \cite{qian1996joint}.
While trying to improve $\Delta t$ by reducing the window width, the $\Delta \upomega$ also deteriorates as the window bandwidth increases.
Therefore, it is impossible to obtain arbitrarily small $\Delta t$ and $\Delta \upomega$ at a time, which is mathematically expressed by the rule
\begin{equation}
    \Delta t \cdot \Delta \upomega  \geq \frac{1}{2}.
\end{equation}
The equality $\Delta t \cdot \Delta \upomega  = \frac{1}{2}$ holds only for a time window $\upgamma(t)$ being a Gaussian curve \cite{qian1996joint}.
To find the proper time window width for computing STFT on a given signal, the Renyi entropy measure can be used \cite{renyi_entropy}. This method involves finding the minimum of the entropy function of the STFT result, depending on the window width, which can require significant computational effort.